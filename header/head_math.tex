% =====================================================================
% === LaTeX Header: Math ==============================================
% =====================================================================
% ---
% ---
% ---

% === Packages ========================================================

\usepackage{amsmath}
\usepackage{amssymb}
\usepackage{amsfonts}
\usepackage{amsthm}

\usepackage{mathtools}
\usepackage{stmaryrd}
\usepackage{nicefrac}

% === TikZ ============================================================

\usepackage{tikz}
  \usetikzlibrary{matrix}
  \usetikzlibrary{fit}
  \usetikzlibrary{backgrounds}
  \usetikzlibrary{arrows}
  \usetikzlibrary{shapes}
  \usetikzlibrary{calc}
  \usetikzlibrary{positioning}
  \usetikzlibrary{mindmap}

% --- Stuff

\usepackage{calc}
\usepackage{ifthen}

% === Theorem Styles ==================================================

\newtheoremstyle{thm_slanted}
{8pt} %Space above
{8pt} %Space below
{\slshape}    %Body font
{} % Indent amount 
{\bfseries}  % Thm head font
{}           % Punctuation after thm head
{0.5em}        % Space after thm head
{\thmname{#1}\thmnumber{ [#2]}\thmnote{ (#3)}}

\newtheoremstyle{thm_normal}
{8pt} %Space above
{8pt} %Space below
{}    %Body font
{} % Indent amount 
{\bfseries}  % Thm head font
{}           % Punctuation after thm head
{0.5em}        % Space after thm head
{\thmname{#1}\thmnumber{ [#2]}\thmnote{ (#3)}}

\newtheoremstyle{thm_slanted_nonumber}
{8pt} %Space above
{8pt} %Space below
{\slshape}    %Body font
{} % Indent amount 
{\bfseries}  % Thm head font
{}           % Punctuation after thm head
{0.5em}        % Space after thm head
{\thmname{#1}\thmnote{ (#3)} }

\newtheoremstyle{thm_normal_nonumber}
{8pt} %Space above
{8pt} %Space below
{}    %Body font
{} % Indent amount 
{\bfseries}  % Thm head font
{}           % Punctuation after thm head
{0.5em}        % Space after thm head
{\thmname{#1}\thmnote{ (#3)} }

% === Theorem Proofs ==================================================

% Kein Punkt am Ende von „Beweis.“
\makeatletter
\let\@addpunct\@gobble
\makeatother

\renewcommand{\qedsymbol}{$\Box$}

% Doppelpunkt am Ende von „Beweis:“
\addto{\captionsngerman}{%
  \def\proofname{Beweis:}
}

% === Theorem Environments ============================================

\theoremstyle{thm_slanted}
  \newtheorem{satz}{Satz}[section]
  \newtheorem{hsatz}[satz]{Hauptsatz} 
  \newtheorem{bem}[satz]{Bemerkung} 
  \newtheorem{folg}[satz]{Folgerung}

  \newtheorem{prop}[satz]{Proposition} 
  \newtheorem{thrm}[satz]{Theorem}
  \newtheorem{lemm}[satz]{Lemma}
  \newtheorem{kor}[satz]{Korollar}

  \newtheorem{defin}[satz]{Definition}

\theoremstyle{thm_normal}
  \newtheorem*{anm}{\emph{Anmerkung}}
  \newtheorem*{frage}{Fragestellung}
  \newtheorem{alg}[satz]{Algorithmus}
  \newtheorem{bsp}[satz]{Beispiel}
  \newtheorem{aufg}{Aufgabe}

\theoremstyle{thm_slanted_nonumber}
  \newtheorem{satzn}{Satz}
  \newtheorem{hsatzn}{Hauptsatz} 
  \newtheorem{bemn}{Bemerkung} 
  \newtheorem{folgn}{Folgerung}

  \newtheorem{propn}{Proposition} 
  \newtheorem{thmn}{Theorem}
  \newtheorem{korn}{Korollar}
  \newtheorem{lemmn}{Lemma}

  \newtheorem{definn}{Definition}

\theoremstyle{thm_normal_nonumber}
  \newtheorem{algn}{Algorithmus}
  \newtheorem{aufgn}{Aufgabe}
  \newtheorem{bspn}{Beispiel}

% === Aliases / Operators =============================================

% --- Numbers

\newcommand{\NN}{\mathbb N}
\newcommand{\ZZ}{\mathbb Z}
\newcommand{\QQ}{\mathbb Q}
\newcommand{\RR}{\mathbb R}
\newcommand{\CC}{\mathbb C}

% --- Fields

\newcommand{\KK}{\mathbb K}
\newcommand{\FF}{\mathbb F}

% --- Sets

\newcommand{\LL}{\mathbb L}

\newcommand{\sbeq}{\subseteq}
\newcommand{\speq}{\supseteq}

\newcommand{\sbneq}{\subsetneq}
\newcommand{\spneq}{\supsetneq}

\newcommand{\idealin}{\trianglelefteq}
\newcommand{\idealni}{\trianglerighteq}

% --- Script

\newcommand{\BB}{\mathcal B}
\newcommand{\MM}{\mathcal M}
\newcommand{\TT}{\mathcal T}

% --- Brackets

\def\<{\left\langle}
\def\>{\right\rangle}

% --- Arrows

\newcommand{\ra}{\rightarrow}
\newcommand{\lra}{\longrightarrow}
\newcommand{\mt}{\mapsto}
\newcommand{\lmt}{\longmapsto}

\newcommand{\Ra}{\Rightarrow}
\newcommand{\La}{\Leftarrow}

\newcommand{\gdw}{\Leftrightarrow}

\newcommand{\col}{\colon}

% --- Definitions

\newcommand{\ceq}{\coloneqq}
\newcommand{\qec}{\eqqcolon}

% --- Foundations

\DeclareMathOperator{\Pot}{Pot}

\renewcommand{\Im}{\text{\textup{Img}}}
\DeclareMathOperator{\Ker}{Ker}

\DeclareMathOperator{\id}{id}

% --- Analysis

\newcommand{\del}{\partial}
\DeclareMathOperator{\dd}{d}

\DeclareMathOperator{\Const}{Const}

% --- Algebra

\DeclareMathOperator{\Quot}{Quot}

\DeclareMathOperator{\Char}{Char}

\DeclareMathOperator{\Coker}{koker}

\DeclareMathOperator{\End}{End}
\DeclareMathOperator{\Hom}{Hom}
\DeclareMathOperator{\Mat}{Mat}
\DeclareMathOperator{\GL}{GL}
\DeclareMathOperator{\Eig}{Eig}

\DeclareMathOperator{\Der}{Der}
\def\IDer{\mathbb{D}\text{\textup{er}}}

% --- Self made

\newcommand{\quotspace}[2]
  {{\raisebox{.1em}{$#1$}\hspace{-0.15em}\left/\raisebox{-.15em}{$#2$}\right.}}

\newcommand{\polyn}[3]
  { {#1}^{#3} + {#2}_{{#3} -1}{#1}^{{#3} -1} + \dots + {#2}_0 }

\newcommand{\laurent}[1]
  { (\kern-0.35ex( #1  )\kern-0.35ex) }


























