\section{Vorwort}

Dieser Vortrag entsand im Rahmen des mathematischen Vorkurses der
Fachschaft MathPhys an der Universität Heidelberg und wurde zum ersten Mal
im Wintersemester 2012/13 gehalten. Da dieser Vortrag, wie der gesamte
Vorkurs, stets weiterentwickelt und verbessert wird, bitte ich ausdrücklich
darum, mir Fragen, Anmerkungen und Korrekturen
zukommen\footnote{\href{mailto:fabian@mathphys.fsk.uni-heidelberg.de}{fabian
    @ mathphys.fsk.uni-heidelberg.de}} zu lassen.

\subsection*{Notation}

Wir bezeichnen die Menge der natürlichen Zahlen mit $\NN \ceq \{
1,2,3,\dots \}$ und meinen mit $\NN_{0} \ceq \NN\cup\{0\}$ die Menge der
natürlichen Zahlen zusammen mit der Null. Ferner bezeichnen wir mit $\ZZ$
die Menge der ganzen Zahlen, mit $\QQ$ die Menge der rationalen Zahlen und
mit $\RR$ die Menge der reellen Zahlen. Wir werden diese Zahlenmengen nicht
rigoros definieren, sondern vertrauen auf das intuitive Verständnis des
Lesers oder der Leserin.
