\subsection{Eine erste Strukturaussage}

% ---

Wir werden an dieser Stelle eine Strukturaussage über Abbildungen
zeigen. In der Tat lässt sich jede Abbildung als Verkettung einer
injektiven, einer surjektiven und einer bijektiven Abbildung schreiben. Um
diese Aussage zu beweisen, wenden wir sämtliche Erkenntnisse aus diesem
Vortrag mehr oder weniger explizit an.

% ---

\begin{bem}
\label{bem:niveau}

  Seien $X,Y$ Mengen und $f\col X \lra Y$ eine Abbildung. Dann definiert
  \begin{align*}
    x_{1} \sim x_{2} \quad :\gdw \quad f(x_{1}) = f(x_{2})
  \end{align*}
  eine Äquivalenzrelation auf $X$.
  \begin{proof}
    Dies folgt sehr einfach aus den Eigenschaften der „$=$“-Relation.
  \end{proof}
\end{bem}

% ---

\begin{satz}

  Seien $X,Y$ Mengen und $f\col X \lra Y$ eine Abbildung. Wir betrachten das
  Diagramm

  \begin{center}
    \begin{tikzpicture}
      \matrix(m)[
      matrix of math nodes,
      row sep=4em,
      column sep=4em,
      text height=2ex, 
      text depth=0.5ex
      ]
      { 
        X & Y \\
        \quotspace{X}{\sim} & \Im(f) \\
      };
      \path[->,font=\scriptsize,>=angle 90]
      (m-1-1) edge node[above] {$f$} (m-1-2)
      (m-1-1) edge node[left] {$\pi$} (m-2-1)
      (m-2-1) edge node[below] {$\varphi$} (m-2-2)
      (m-2-2) edge node[right] {$\iota$} (m-1-2);
    \end{tikzpicture}
  \end{center}
  
  \noindent wobei $\pi\col x \mt [x]$ die Projektion bzgl. der Relation aus
  [\ref{bem:niveau}] ist und $\iota$ die natürliche Inklusion von
  $f(X)\sbeq Y$ ist. Weiter sei
  \begin{align*} 
    \varphi\col \quotspace{X}{\sim} &\lra \Im(f) \\
    [x] &\lmt f(x)
  \end{align*}
  Dann gilt
  \begin{enumerate}
    \item $\varphi$ ist wohldefiniert.
    \item $\varphi$ ist bijektiv.
    \item Das Diagramm kommutiert in dem Sinne, dass $f = \iota\circ \varphi
      \circ \pi$ gilt.
  \end{enumerate}

  \begin{proof}
  \quad
  
  \begin{enumerate}

    \item Seien $x_{1},x_{2}\in X$, mit $x_{1}\sim x_{2}$. Dann gilt
      nach Definition von $\sim$, dass $f(x_{1}) = f(x_{2})$. Also
      \begin{align*}
        \varphi\big([x_{1}]\big) = f(x_{1}) = f(x_{2}) = \varphi\big([x_{2}]\big).
      \end{align*}

    \item Wir zeigen die Injektivität und die Surjektivität von
      $\varphi$.
      \begin{itemize}

        \item[\tiny{(inj.)}] Seien $x_{1},x_{2}\in X$ mit
          $\varphi\big([x_{1}]\big) = \varphi\big([x_{2}]\big)$. Dies bedeutet
          $f(x_{1}) = f(x_{2})$ und damit ist $x_{1}\sim x_{2}$. Nach
          Bemerkung [\ref{bem:vertreter}] ist dann $[x_{1}] = [x_{2}]$.
        
        \item[\tiny{(surj.)}] Sei $y\in f(X)$. Dann existiert nach
          Definition von $f(X)$ ein $x\in X$ mit $f(x) = y$. Betrachten wir
          die Äquivalenzklasse von $x$, dann sehen wir
          $\varphi\big([x]\big) = f(x) = y$.

      \end{itemize}

    \item Wir zeigen die Gleichheit der Abbildungen
      \begin{align*}
        f &\col X \lra Y \\
        \iota\circ\varphi\circ\pi &\col X \lra Y
      \end{align*}
      Diese Abbildungen haben offensichtlich die selben Definitionsbereiche
      und Zielmengen. Bleibt noch zu zeigen, dass sie punktweise
      übereinstimmen. Sei also $x\in X$. Wir berechnen
      \begin{align*}
        (\iota\circ\varphi\circ\pi)(x) = (\iota\circ\varphi)(\pi(x)) =
        (\iota\circ\varphi)\big([x]\big) = \iota\Big(\varphi\big([x]\big)\Big) =
        \iota(f(x)) = f(x).
      \end{align*}
    \end{enumerate}
    Damit ist alles gezeigt.
    
  \end{proof}

\end{satz}

% ---
