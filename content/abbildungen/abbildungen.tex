\section{Abbildungen}
\label{sec:abb}

Im letzten Kapitel haben wir den Begriff der \emph{Relation} eingeführt, sind
aber schnell zum Studium der spezielleren Äquivalenzrelationen übergegangen. Wir
wollen uns in diesem Kapitel mit Abbildungen (oder Funktionen) beschäftigen. Um
diese jedoch mathematisch korrekt formulieren zu können, brauchen wir einen
allgemeineren Begriff der \emph{Relation}, denn eine Abbildung setzt Elemente
\emph{verschiedener} Mengen in Beziehung. Die Äquivalenzrelationen aus dem
vorherigen Kapitel waren begrifflich nur in der Lage, Elemente aus \emph{ein und
  derselben} Menge in Beziehung zu setzen.

\begin{defin}
  Seien $X,Y$ Mengen und $R \sbeq X\times Y$. Dann nennen wir $R$ eine
  \emph{Relation zwischen $X$ und $Y$}.
\end{defin}

Auch hier schreiben wir für $(x,y) \in R$ einfach $x \sim y$, falls keine
Verwechlungsgefahr besteht. Inbesondere sprechen wir nun von allgemeinen
Relationen und nicht von Äquivalenzrelationen. Die hier betrachteten Relationen
müssen nicht länger reflexiv, symmetrisch oder transitiv sein; sie können diese
Eigenschaften meist garnicht besitzen.

Wir werden uns in diesem Kapitel dem Abbildungsbegriff der Hochschulmathematik
nähern. Diese Näherung erfolgt in drei Schritten. Im ersten Schritt werden wir
Abbildungen mit Blick auf ihren \emph{Zuordnungscharakter} betrachten. Dazu
werden wir den bekannten Funktionenbegriff aus der Schulmathematik aufgreifen
und in eine mengentheoretisch saubere Formulierung überführen. Außerdem erwähnen
wir kurz zwei wichtige Spezialfälle von Zuordnungen. Im zweiten Schritt
fokussieren wir das Änderungsverhalten bzw. den \emph{Kovariationsaspekt} von
Abbildungen. Im Sinne eines Ausblicks auf die zentralen Begriffe der
Grundvorlesungen „Analysis 1“ und „Lineare Algebra 1“ werden wir
\emph{Stetigkeit} und \emph{Linearität} als Kovariationseigenschaft von
Abbildungen motivieren. Im letzten Schritt betrachten wir Abbildungen als
eigenständige mathematische Objekte.

% ---

\subsection{Abbildungen als Zuordnungsrelationen}

% ---

Aus der Schule kennen wir (so oder so ähnlich) den folgenden Begriff einer
Abbildung.

\begin{definn}
  Eine \emph{Abbildung} $f$ ist eine Zuordnungsvorschrift, die
  jedem Element $x$ genau ein Element $y$ zuordnet.
\end{definn}

Schrecklich, nicht wahr? Was meinen wir mit \emph{zuordnen}, was soll denn eine
\emph{Zuordnungsvorschrift} sein; was sind $x$ und $y$? Diese Definition ist
total schwammig. Zum Glück können wir bereits mit unseren Mitteln einen exakten
Abbildungsbegriff definieren. Als Motivation dafür dient der aus der
Schulmathematik bekannte \emph{Funktionsgraph}.

% ---

\begin{center}
  \includegraphics[width=0.75\textwidth]{./material/graph.png}
\end{center}

% ---

In diesem Sinne ordnet eine Abbildung einem Element $x$ der horizontalen Achse
genau dann ein Element $y$ der vertikalen Achse zu, falls der Punkt $(x, y)$ auf
dem Funktionsgraphen liegt. Interpretieren wir $(x, y)$ nun nicht länger als
geometrischen Punkt, sondern als Paar von Elementen, so nähern wir uns einer
Definitionsmöglichkeit von Abbildungen -- ausgehend von dem Relationenbegriff.
Im abstrakten Setting werden wir in der Hochschulmathematik nicht nur
Abbildungen zwischen den reellen Zahlen studieren, sondern allgemein Abbildungen
zwischen beliebigen Mengen. Wir wagen einen ersten Definitionsversuch.

% ---

\begin{definn}
  Seien $X,Y$ Mengen und $f \sbeq X\times Y$. Dann nennen wir $f$ eine
  \emph{Abbildung von $X$ nach $Y$}. Falls für $x\in X$ und $y\in Y$ gilt, dass
  $(x, y) \in f$, dann schreiben wir auch $f(x) = y$ oder $f\col x \lmt y$.
\end{definn}

% ---

Dem aufmerksamen Leser wird auffallen, dass sich diese Definition (bis auf die
etwas andere Schreibweise) nicht von der Definition einer \emph{Relation}
unterscheidet (siehe Anfang des Kapitels). Dem pfiffigen Leser wird darüber
hinaus auffallen, dass wir nicht alle Aspekte der „Schuldefinition“ einer
Abbildunge berücksichtigt haben. Bisher haben wir nur das \emph{Zuordnen}
formalisiert.

Die Forderung, dass „jedem Element $x$ genau ein Element $y$ [zugeordnet wird]“,
ist in unserem Definitionsversuch nicht enthalten. Sie besteht aus zwei Teilen:
Jedes zulässige Element $x$ „abbildbar“ sein und die Zuordnung zu $y$ muss
eindeutig sein. Die zu diesen Eigenschaften passenden Begriffe der
Hochschulmathematik lauten \emph{linkstotal} und \emph{rechtseindeutig}.

% ---

\begin{defin}
  Seien $X,Y$ Mengen und $R \sbeq X\times Y$ eine Relation zwischen $X$ und $Y$.
  Wir nennen $R$ \emph{linkstotal}, falls für jedes Element $x\in X$ ein $y\in
  Y$ existiert, sodass $x \sim y$.
  %% \emph{rechtstotal}, falls für jedes Element $y\in Y$ ein
  %% $x\in X$ existiert, sodass $x \sim y$.
\end{defin}

% ---

Eine Relation heißt demnach \emph{linsktotal}, falls wir für jedes Element aus der
\emph{linken} Menge $X$ des kartesischen Produkts $X\times Y$ einen Partner aus
$Y$ finden. In der Sprache der Abbildungen gesprochen: Jedes Element aus $X$ ist
eine zulässige Eingabe, $X$ ist der Definitionsbereich der Abbildung.

% ---

\begin{defin}
  Seien $X,Y$ Mengen und $R \sbeq X\times Y$ eine Relation zwischen $X$ und $Y$.
  Wir nennen $R$ \emph{rechtseindeutig}, falls zu jedem $x\in X$ höchstens ein
  $y\in Y$ existiert, sodass $x\sim y$.
  %% \item \emph{linkseindeutig}, falls zu jedem $y\in Y$ höchstens ein
  %% $x\in X$ existiert, sodass $x\sim y$
\end{defin}

% ---

Eine Relation heißt demnach \emph{rechtseindeutig}, falls wir für jedes Element
aus $X$ maximal ein Element in der \emph{rechten} Menge des kartesischen
Produkts $X\times Y$ finden. In diesem Sinne wird das Wurzelziehen
$\sqrt{\cdot}$ keine Abbildung sein, da für $\sqrt{4}$ eben die Zahlen $2$ und
$-2$ in Frage kommen und so sowohl $4 \sim 2$ und $4 \sim (-2)$ gelten.

% ---

Wir haben nun sämtlichen Handwerkszeug beisammen, um eine mengentheoretisch
saubere Definition von \emph{Abbildungen} im abstrakten Sinne anzugeben. Wir
ordnen hierfür einige Begriffe neu an und kodieren in unserem Abbildungsbegriff
insgesamt drei Informationen: den Definitionsbereich, die Zielmenge und den
Graph als spezielle Relation zwischen Definitionsbereich und Zielmenge.

\begin{defin}
  Seien $X,Y$ Mengen und $G \sbeq X\times Y$ eine Relation zwischen $X$ und
  $Y$. Ist $G$ linkstotal und rechtseindeutig, so nennen wir das Tripel
  $f\ceq (X,Y,G)$ eine \emph{Abbildung von $X$ nach $Y$} und schreiben
  \begin{align*}
    f \col X \lra Y.
  \end{align*}
  Es gilt also:
  \begin{itemize}
    \item zu jedem $x\in X$ existiert ein $y \in Y$ mit $x \sim y$
    \item und dieses $y$ ist das einzige Element in $Y$ mit $x \sim y$.
  \end{itemize}
  Dieses zu $x$ eindeutige $y$ mit $x\sim y$ bezeichnen wir mit
  $f(x)$ und schreiben
  \begin{align*}
    f(x) = y \,\text{ oder }\, f\col x \lmt y.
  \end{align*}
  Gegebenfalls nennen wir $X$ den \emph{Definitionsbereich}, $Y$ die
  \emph{Zielmenge} und $G$ den \emph{Graph} der Abbildung $f$.
\end{defin}

% ---

Dies ist die „richtige“ Definition des Abbildungsbegriffs. Alle Grundbegriffe,
die wir in der Definition verwendet haben, sind bekannt und eindeutig definiert.
Darüber hinaus haben wir nun eine formal saubere Fassung des
Zuordnungscharakters der Schuldefinition.

Für die Hochschulmathematik werden mit Blick auf den Zuordnungsaspekt von
Abbildngen zwei Begriffe relevant, die in der Schulmathematik kaum Beachtung
finden.

% ---

\begin{defin}
  Seien $X,Y$ Mengen und $f\col X \lra Y$ eine Abbildung. 
  \begin{enumerate}
    \item Wir nennen $f$ \emph{injektiv}, falls für alle $x_{1}, x_{2}\in X$ mit
      $x_{1} \neq x_{2}$ auch $f(x_{1}) \neq f(x_{2})$ gilt.
    \item Wir nennen $f$ \emph{surjektiv}, falls für alle $y\in Y$ ein $x\in X$
      existiert, mit $f(x)=y$.
  \end{enumerate}
\end{defin}

% ---

Unter \emph{injektiven} Abbildungen verstehen wir Zuordnungen, die
Ungleichheiten oder Trennungen erhalten. Starte ich mit zwei Elementen, etwa die
Zahlen $2$ und $-2$, von denen ich weiß, dass sie unterschiedlich sind, dann
soll eine injektive Abbildung diese Elemente auch auf unterschiedliche Elemente
Abbilden. Die Abbildung $f\col \RR \lra \RR; x \lmt x^{2}$ wäre demnach keine
injektive Abbildung, da $f(-2) = 4 = f(2)$ gilt.

Unter \emph{surjektiven} Abbildungen verstehen wir Zuordnungen, die ihre gesamte
Zielmenge ausschöpfen. Für jedes Element des Zielbereichs müssen wir ein Element
aus dem Definitionsbereich finden, dass Ersterem zugeordnet wird. Betrachten wir
wieder die Abbildung $f\col \RR \lra \RR; x \lmt x^{2}$. Diese ist nicht
surjektiv, da echte Quadrate immer positiv sind wir demnach für die Zahl $-2$
kein Urbild finden.

Die aus mengentheoretischer Sich kanonischen Beispiele für injektive und
surjektive Abbildungen sind \emph{Inklusionen} und \emph{Projektionen}. 

% ---

\begin{defin}
  Seien $A,X$ Mengen mit $A\sbeq X$. Dann nennen wir
  \begin{align*}
    \iota\col A &\lra X \\
    a &\lmt a
  \end{align*}
  die \emph{natürliche Inklusion} von $A$ in $X$
\end{defin}

\begin{anm}
  $\iota$ ist eine injektive Abbildung.
\end{anm}

% ---

\begin{defin}
  Sei $X$ eine Menge und $\sim$ ein Äquivalenzrelation auf $X$. Dann nennen
  wir
  \begin{align*}
    \pi \col X &\lra \quotspace{X}{\sim} \\
    x &\lmt [x]
  \end{align*}
  die \emph{natürliche Projektion} bzgl. $\sim$.
\end{defin}
% ---

\begin{anm}
  $\pi$ ist eine surjektive Abbildung.
\end{anm}

% ---

\begin{defin}
  Sei $X$ eine Menge. Dann nennen wir
  \begin{align*}
    \id_{X}: X &\lra X \\
    x &\lmt x
  \end{align*}
  die \emph{Identität} auf $X$.
\end{defin}

% ---

\begin{anm}
  $\id_{X}$ ist eine gleichzeitig injektiv und surjektiv Abbildung.
\end{anm}

% ---

Diese eben definierten Eigenschaften von Abbildungen, tauchen in der Mathematik
immer wieder auf. In der Tat sind die drei genannten Beispiele Inklusion,
Projektion und Identität \emph{die} zentralen Beispiele für injektive und
surjektive Abbildungen. Wir können uns jede injektive Abbildung als Inklusion
vorstellen und jede surjektive Abbildung als eine Projektion auf
Äquivalenzklassen interpretieren. Vermutlich kann man diese Erkenntnis erst dann
voll und ganz begreifen und würdigen, falls man am Ende des ersten Jahres des
Mathematikstudiums darauf zurückblickt. Schreibt Euch am besten eine Erinnerung
in Euren Kalender: „TODO: Alle injektiven und surjektiven Abbildung, die ich im
letzten Jahr kennengelern habe, als Inklusion oder Projektion interpretieren“.

Will man die Eigenschaft einer Abbildung, injektiv oder surjektiv zu sein,
nachzuweisen, gibt es verschiedene Möglichkeiten. Die Gültigkeit einiger dieser
Möglichkeiten werden wir nachfolgend zeigen. Insbesondere stellen wir den
Zusammenhang zu der Begriffen der Totalität und Eindeutigkeit von Relationen
her. Für den Abbildungsbegriff haben wir nur von Linkstotalität und
Rechtseindeutigkeit gesprochen. Tatsächlich verbergen sich hinter den parallel
formulierten Begriffen der Rechtstotalität und Linkseindeutigkeit gerade die
Begriffe der Surjektivität und Injektivität.

\begin{defin}
  Seien $X,Y$ Mengen und $R \sbeq X\times Y$ eine Relation zwischen $X$ und $Y$.
  \begin{itemize}
  \item Wir nennen $R$ \emph{rechtstotal}, falls für jedes Element $y\in Y$ ein
    $x\in X$ existiert, sodass $x \sim y$.
  \item Wir nennen $R$ \emph{linkseindeutig}, falls zu jedem $y\in Y$ höchstens
    ein $x\in X$ existiert, sodass $x\sim y$
  \end{itemize}
\end{defin}

% ---

\begin{bem}
\label{bem:inj}

  Seien $X, Y$ Mengen und $f\col X \lra Y$ eine Abbildung. Dann sind
  die folgenden Aussagen äquivalent.
  \begin{enumerate}

    \item $f$ ist injektiv, d.h. für alle $x_{1}\neq x_{2} \in X$ gilt
      $f(x_{1})\neq f(x_{2})$.

    \item Für alle $x_{1},x_{2}\in X$ mit $f(x_{1})=f(x_{2})$ gilt
      bereits $x_{1}=x_{2}$.

    %% \item Für alle $y\in Y$ ist $f^{-1}(\{ y \})$ höchstens
    %%   einelementig.

    \item Der Graph $G$ von $f$ ist eine linkseindeutige Relation.

  \end{enumerate}

  \begin{proof}
    \quad

    \begin{itemize}
    \item[\tiny{(i) $\Ra$ (ii)}] 
      Dies folgt direkt durch Kontraposition.

    %% \item[\tiny{(ii) $\Ra$ (iii)}] Sei $y\in Y$, $f^{-1}(\{ y \}) \neq
    %%   \emptyset$ und $x_{1},x_{2}\in f^{-1}(\{ y \})$. Dann gilt $f(x_{1})
    %%   = y = f(x_{2})$, also wegen (ii) bereits $x_{1}=x_{2}$. Also ist
    %%   $f^{-1}(\{ y \}) = \{ x_{1} \}$.

    \item[\tiny{(ii) $\Ra$ (iii)}] Seien $y\in Y$ und $x_{1},x_{2}\in X$ mit
      $x_{1}\sim_{G}y$ und $x_{2}\sim_{G}y$. Dies bedeutet gerade, dass $y =
      f(x_{1}) = f(x_{2})$. Also gilt wegen (ii) bereits $x_{1} = x_{2}$. Es ist
      demnach $\sim_{G}$ linkseindeutig.

    \item[\tiny{(iii) $\Ra$ (i)}] Seien $x_{1}, x_{2} \in X$ mit $x_{1}\neq
      x_{2}$. Angenommen es gilt $f(x_{1}) = f(x_{2}) \qec y$. Dann gilt aber
      $x_{1}\sim_{G}y$ und $x_{2}\sim_{G} y$. Dies ist ein Widerspruch zur
      Linkseindeutigkeit von $\sim_{G}$.

    \end{itemize}

  \end{proof}

\end{bem}

% ---

\begin{bem}

  Seien $X, Y$ Mengen und $f\col X \lra Y$ eine Abbildung. Dann sind
  die folgenden Aussagen äquivalent.

  \begin{enumerate}

    \item $f$ ist surjektiv, d.h. für alle $y\in Y$ existiert ein $x\in
      X$, sodass $f(x)=y$.

    %% \item Für alle $y\in Y$ ist $f^{-1}(\{ y \})$ mindestens
    %%   einelementig.

    %% \item Es gilt $f(X) = Y$.

    \item Der Graph $G$ von $f$ ist rechtstotale Relation.

  \end{enumerate}

  \begin{proof}
    \quad

    \begin{itemize}
    
      \item[\tiny{(i) $\Ra$ (ii)}] Sei $y\in Y$, dann existiert ein $x\in X$ mit
        $f(x)=y$. Per Definition bedeutet das, dass $(x, y) \in G$. Das
        funktioniert für jedes $y\in Y$, also ist der Graph rechtstotal.

      %% \item[\tiny{(ii) $\Ra$ (iii)}] Sei $y\in Y$, dann ist $f^{-1}(\{ y
      %%   \})$ mindestens einelementig. Also existiert ein $x\in X$ mit $x\in
      %%   f^{-1}(\{ y \})$, d.h. $f(x) = y$. Also ist $y \in f(X)$, was
      %%   „$\speq$“ zeigt. Die Inklusion „$\sbeq$“ gilt nach Definition der
      %%   Bildabbildung.

      %% \item[\tiny{(ii) $\Ra$ (iii)}]
      %%   Sei $y\in Y$. Wir zeigen die Existens von einem Element $x\in X$,
      %%   mit $x\sim_{G} y$. Da $f(X) = Y$ gilt $y\in f(X)$, also existiert
      %%   ein $x\in X$ mit $f(x)= y$. Mit anderen Worten $x\sim_{G}y$.

      \item[\tiny{(ii) $\Ra$ (i)}] Sei $y\in Y$. Da $\sim_{G}$ rechtstotal ist,
        existiert ein $x\in X$ mit $x\sim_{G} y$. Anders formuliert: $f(x) = y$.
        Wir finden also für jedes vorgelegte $y \in Y$ ein Urbild.

    \end{itemize}
    
  \end{proof}

\end{bem}

% ---
