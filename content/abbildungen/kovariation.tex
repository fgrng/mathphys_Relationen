% ---

\subsection{Abbildungsbegriffe mit Änderungsaspekt}

% ---

Wir wollen in diesem Abschnitt zwei zentrale Begriffe aus den Grundvorlesungen
ansprechen. Es geht hierbei nicht darum, diese Begriffe zu verstehen. Vielmehr
wollen wir durch diese Begriffe einen anderen Blickwinkel auf Abbildung
gewinnen.

Wir fokussieren bei Abbildung hier nicht länger die Zuordnung von Argumenten zu
ihren Bildern, sondern werfen einen Blick darauf, wie man Beziehungen von
Änderungsverhalten im Argument einer Funktion (Variation) und Änderungsverhalten
im zugehörigen Bild der Funktion (Kovariation\footnote{Quasi, Variationen in der
  Oppositkategorie, gnihihi. Diese Fußnote ist nur ein Versuch die Tutorinnen
  und Tutoren zu nerdsnipen (siehe \url{https://xkcd.com/356/}). Liebe Erstis,
  ignoriert das hier einfach ;).}) beschreiben kann.

% ---

Wir beschränken uns für den Exkurs in die Vorlesung „Analysis 1“ auf Abbildungen
zwischen reellen Zahlen. Wir setzen dafür die Rechenregeln der reellen Zahlen
und das Absolutbetrags voraus.

\begin{defin}
  Es sei $f\col \RR \lra \RR$ eine Abbildung und $x_{0}\in\RR$ ein Element des
  Definitionsbereichs. Wir nennen $f$ in dem Punkt $x_{0}$ \emph{stetig}, falls
  für jede beliebige Zahl $\varepsilon > 0$ die folgende Aussage gilt: Es existiert
  eine Zahl $\delta > 0$, sodass für alle Elemente $x\in \RR$ im
  Definitionsbereich mit der Eigenschaft $|x - x_{0}| < \delta$ gilt, dass
  $|f(x) - f(x_{0})| < \varepsilon$.
\end{defin}

% ---

Die Definition sieht auf den ersten Blick sehr verwirrend aus. Wenn wir sie
jedoch Stück für Stück auseinandernehmen und als eine Aussage über das
Änderungsverhalten der Funktion interpretieren, wird ein Schuh draus.

% ---

\begin{center}
  \includegraphics[width=0.75\textwidth]{./material/stetig.png}
\end{center}

% ---

Die Definition der Stetigkeit einer Funktion in einem Punkt $x_0$ beginnt mit
der Vorgabe eines Abstands $\varepsilon > 0$ auf der $Y$-Achse für bestimmte
Funktionswerte um $f(x_0)$ (siehe „…, dass $|f(x)~-~f(x_0)|~<~\varepsilon$“ und
den roten \emph{Epsilonschlauch} in der Visualisierung). Es wird bei
vorliegendem $\varepsilon > 0$ gefordert, dass wir einen Abstand $\delta > 0$
auf der $X$-Achse finden (siehe „… mit der Eigenschaft $|x - x_{0}| < \delta$
gilt …“ und die blauen Linien in der Visualisierung), sodass sich die
Funktionswerte von Elementen innerhalb dieses Abstands um $x_0$ innerhalb des
Epsilonschlauchs befinden (siehe die schwarzen Markierungen auf den Schaubild
der Funktion bzw. die dicke grüne Linie).

Zur Verdeutlichung dieses Konzepts betrachten wir eine nicht-stetige Funktion
reeller Zahlen. Die Funktion aus dem folgenden Schaubild ist an der markierten
Stelle nicht stetig.

% ---

\begin{center}
  \includegraphics[width=0.75\textwidth]{./material/nonstetig.png}
\end{center}

% ---

Wir können den (punktweisen) Stetigkeitsbegriff als eine Aussage über ein
\emph{gutartiges} Änderungsverhalten von Abbildungen interpretieren. Sei eine
Abbildung $f$ in einem Punkt $x_0$ stetig. In diesem Fall können wir davon
ausgehen, dass sich die Urbilder von Punkten, die sich in der Nähe von $f(x_0)$
befinden, in der Nähe von $x_0$ finden lassen. Ein in diesem Sinne
\emph{bösartiges}, also plötzliches oder sprunghaftes, Änderungsverhalten äußert
sich nur in Unstetigkeitsstellen (siehe zweite Visualisierung).

Wir beginnen nun einen Exkurs in die Vorlesung „Lineare Algebr 1“ und beziehen
uns auch hier auf Abbildungen zwischen reellen Zahlen.

\begin{defin}
  Es sei $f\col \RR \lra \RR$ eine Abbildung. Wir nennen $f$ eine \emph{lineare}
  Abbildung, falls für alle Elemente $x, y \in \RR$ aus dem Definitionsbereich
  und beliebige Zahlen $\lambda$ die folgenden Aussagen gelten.
  \begin{align*}
    f(x + y) = f(x) + f(y) \quad \text{ und } \quad f(\lambda \cdot x) = \lambda \cdot f(x)
  \end{align*}
\end{defin}

Hier ist die Interpretation der Eigenschaft \emph{Linearität} als eine Aussage
über gutartiges Änderungsverhalten fast offensichtlich. Bei linearen Abbildungen
übertragen sich Rechenoperationen in der Argumenten direkt auf Rechenoperationen
in den Funktionswerten. 
