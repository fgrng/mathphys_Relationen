% ---

\subsection{Abbildungen als eigenständige Objekte}

% ---

In diesem Abschnitt nehmen wir Abbildungen als eigenstände Objekte in den Blick
-- statt eine Abbildung \emph{für sich} zu studieren, betrachten wir Abbildungen
\emph{unter sich}. Dazu werden wir klären, was es für Abbildungen heißt
\emph{gleich} oder \emph{unterschiedlich} zu sein, die Einsicht erhalten, dass
eine Abbildung selten alleine lebt und eine erste Möglichkeit kennenlernen, mit
Abbildungen zu \emph{rechnen}.

% ---

\begin{defin}
  Seien $X, Y, A, B$ Mengen, sowie $f\col X \lra Y$, $g\col A\lra B$
  Abbildungen. Dann definieren wir die \emph{Gleichheit} von $f$ und $g$
  via
  \begin{align*}
    f = g \quad :\gdw \; \begin{cases}
      & X = A \\
      \text{ und }  & Y = B \\
      \text{ und }  & f(x) = g (x) \quad \text{für alle } x\in X = A
    \end{cases}
  \end{align*}
\end{defin}

% ---

Im ersten Abschnitt über Abbildungen haben wir in dem Abbildungsbegriff drei
Informationen kodiert: den Definitionsbereich, die Zielmenge und die
Zuordnungsrelation. Um zwei Abbildungen \emph{gleich} nennen zu dürfen, forden
wir, dass alle drei Informationen übereinstimmen.

Dies wird besonders relevant, wenn wir uns mit der \emph{Bildabbildung} und
\emph{Urbildabbildung} beschäftigen. Betrachten wir eine Abbildung $f\col X \lra
Y$ zwischen zwei Mengen, dann erhalten wir automatisch zwei weitere Abbildungen,
die zwar ähnlich heißen, sich jedoch insbesondere in Definittionsbereich und
Zielmenge unterscheiden.

% ---

\begin{defin}
  Seien $X,Y$ Mengen und $f\col X \lra Y$ eine Abbildung. Dann induziert $f$
  zwei weitere Abbildungen zwischen den Potenzmengen $\Pot(X),\Pot(Y)$:
  \begin{align*}
    f \col \Pot(X) &\lra \Pot(Y) \\
    A &\lmt \{ f(a) \in Y \mid a\in A \}
    \intertext{die \emph{Bildabbildung} zu $f$ und}
    f^{-1} \col \Pot(Y) &\lra \Pot(X) \\
    B &\lmt \{ x \in X \mid f(x)\in B \}
  \end{align*}
  die \emph{Urbildabbildung} zu $f$. Wir nennen $f(X)$ das \emph{Bild von
    $f$} und schreiben auch $\Im(f)$.
\end{defin}

% ---

Jetzt, wo wir Abbildungen unterscheiden können, wird es Zeit mit Abbildungen
eigene mathematische Operationen durchzuführen. Die wichtigste Verknüpfung
zweier Abbildung ist deren Hintereinanderausführung.

% ---

\begin{defin}
  Seien $X,Y,Z$ Mengen und $f\col X \lra Y$, sowie $g\col Y \lra Z$
  Abbildungen. Dann definieren wir die Abbildung
  \begin{align*}
    g \circ f \col X &\lra Z \\
    x &\lmt g(f(x))
  \end{align*}
  (gelesen: „$g$ nach $f$“) als die \emph{Komposition} oder
  \emph{Verkettung} von $f$ und $g$.
\end{defin}

% ---

\begin{defin}

  Es seien $X,Y$ Mengen und $f\col X \lra Y$, $g\col Y \lra X$
  Abbildungen.

  \begin{enumerate}
  \item Wir nennen $g$ eine \emph{Linksumkehrung} zu $f$, falls
    $g\circ f = \id_{X}$ gilt.
  \item Wir nennen $g$ eine \emph{Rechtsumkehrung} zu $f$, falls
    $f\circ g = \id_{Y}$ gilt.
  \item Wir nennen $g$ eine \emph{Umkehrabbildung} zu $f$, falls
    \begin{align*}
      g\circ f = \id_{X} \text{ und } f\circ g = \id_{Y}
    \end{align*}
    gilt.
  \end{enumerate}
  
\end{defin}

Die Möglichkeit, Abbildungen umzukehren, hängt stark mit Eigenschaften
zusammen, die wir bereits in diesem Kapitel betrachtet haben. In der Tat
ist eine Abbildung genau dann links-umkehrbar, wenn sie injektiv. Ob eine
Abbildung rechts-umkehrbar ist, wenn sie surjektiv ist, bleibt eine
Glaubensfrage.

% --

\begin{bem}

  Es seien $X,Y$ Mengen und $f\col X \lra Y$ eine Abbildung. Dann sind die
  folgenden Aussagen äquivalent.

  \begin{enumerate}
    \item $f$ ist injektiv
    \item Es gibt eine Abbildung $g\col Y \lra X$ mit $g\circ f = \id_{X}$.
  \end{enumerate}

  \begin{proof}
    Um die Äquivalnz dieser Aussagen zu zeigen, müssen wir zwei
    Implikationen beweisen.
    \begin{itemize}
    \item[\tiny{(ii) $\Ra$ (i)}] Es gebe eine Abbildung $g\col Y \lra X$
      mit $g\circ f = \id_{X}$. Um zu zeigen, dass $f$ injektiv ist,
      verwenden wir das Kriterium (ii) aus Bemerkung [\ref{bem:inj}].

      Seien also $x_{1}, x_{2} \in X$ mit $f(x_{1}) = f(x_{2})$. Durch
      Anwendung von $g$ auf beiden Seiten dieser Gleichung erhalten wir
      $(g\circ f)(x_{1}) = (g\circ f)(x_{2})$ und wegen der Eigenschaft von
      $g$ demnach $\id_{X}(x_{1})=\id_{X}(x_{2})$. Da die Identität jedes
      Element fest lässt, gilt schließlich $x_{1}=x_{2}$. Dies zeigt, dass
      $f$ injektiv ist.
    \item[\tiny{(i) $\Ra$ (ii)}] Sei $f$ injektiv. Unser Ziel ist es, eine
      Abbildung $g\col Y \lra X$ zu konstruieren, welche $f$ rückgängig
      macht. Dafür ist es uns egal, wie sich $g$ außerhalb des Bildes
      $f(X)$ von $f$ verhält. Wir fixieren also wahllos ein $x_{0}\in X$
      und setzen $g(y) \ceq x_{0}$ für $y\in Y\setminus f(X)$.

      Wir müssen $g$ noch auf $f(X)$ konstruieren. Ist $y\in f(X)$ dann
      existiert nach Definition von $f(X)$ ein Urbild $x_{y}\in X$ mit
      $f(x_{y})=y$. Wegen Bemerkung [\ref{bem:inj}] ist dieses $x_{y}$
      eindeutig. Wir setzen für jedes $y\in f(X)$ das Bild unter $g$ auf
      dieses eindeutige $g(y) \ceq x_{y}$. Damit ist $g$ eine auf ganz $Y$
      wohldefinierte Abbildung. Dieses $g$ ist nach Konstruktion eine
      Linksumkehrung von $f$, denn für $x\in X$ gilt
      \begin{align*}
        (g\circ f)(x) 
        = g (f(x))
        = x_{f(x)}
        = x.
      \end{align*}
    \end{itemize}
  \end{proof}
  
\end{bem}

% ---

\begin{bem}

  Es seien $X,Y$ Mengen und $f\col X \lra Y$ eine Abbildung. Dann sind die
  folgenden Aussagen äquivalent.

  \begin{enumerate}
    \item $f$ ist surjektiv.
    \item Es gibt eine Abbildung $g\col Y \lra X$ mit $f\circ g = \id_{Y}$.
  \end{enumerate}

  \begin{proof}
    \quad 

    \begin{itemize}
    \item[\tiny{(ii) $\Ra$ (i)}] Es gebe eine Abbildung $g\col Y \lra X$
      mit $f\circ g = \id_{Y}$. Um zu zeigen, dass $f$ surjektiv, müssen
      wir für jedes $y\in Y$ ein Urbild in $X$ finden. Es sei $y\in Y$;
      dann ist $g(y)\in X$. Durch die Eigenschaft von $g$ gilt dann
      $f(g(y)) = (f\circ g)(y) = y$, also ist $g(y)$ das gesuchte Urbild
      unter $f$ zu $y$.
    \item[\tiny{(i) $\Ra$ (ii)}] Sei $f$ surjektiv. Unser Ziel ist es, eine
      Abbildung $g\col Y \lra X$ zu konstruieren, für die $f\circ g$ die
      Identität ist. Es sei $y\in Y$. Da $f$ surjektiv ist, existiert ein
      Urbild $x_{y}\in X$ mit $f(x_{y}) = y$. Für jedes $y\in Y$ wählen wir
      ein (nicht eindeutiges) Urbild $x_{y}\in Y$ und konstruieren so durch
      die Setzung $g(y) \ceq x_{y}$ unsere gesuchte Abbildung\footnote{Es
        stellt sich die Frage, ob dies ein vernünftiger
        Konstruktionsprozess ist. Da wir keine Informationen über die
        Mengen $X$ und $Y$ haben, könnten diese auch unendlich groß
        sein. Um $g$ konstruieren zu können, müssen wir also
        \emph{unendlich oft} eine Auswahl -- die Wahl des Urbilds --
        treffen. Die Frage, ob diese Vorgehensweise gerechtfertigt oder
        überzeugend ist, ist eine Glaubensfrage. Das sogenannte
        Auswahlaxiom ermöglicht uns diese Vorgehensweise; seine Gültigkeit
        kann jedoch nicht bewiesen werden.}.
    \end{itemize}
  \end{proof}

\end{bem}

% ---

Existiert für eine Abbildung eine Links- und eine Rechtsumkehrung, dann
sind diese Umkehrungen bereits identisch. Mit den eben gezeigten
Bemerkungen, sehen wir also, dass eine bijektive Abbildung eine eindeutige
beidseitige Umkehrung -- eine Umkehrabbildung -- besitzt.

% ---

\begin{bem}

  Es seien $X, Y$ Mengen und $f\col X\lra Y$. Ferner seien
  $g_{l}, g_{r}\col Y \lra X$ zwei Abbildengen, wobei $g_{l}$ eine
  Linksumkehrung und $g_{r}$ eine Rechtsumkehrung von $f$ ist. Dann gilt
  $g_{l} = g_{r}$.

  \begin{proof}
    Nach den oben formulierten Voraussetzungen habe $g_{l}$ und $g_{r}$ den
    gleichen Wertebereich und die gleiche Zielmenge. Wir müssen noch
    zeigen, dass sie punktweise Übereinstimmen. Es gelten für alle $x\in X$
    und alle $y\in Y$ die Gleichungen
    \begin{align*}
      (g_{l}\circ f)(x) = \id_{X}(x) \qquad \text{ und } \qquad (f\circ g_{r})(y) = y.
    \end{align*}
    Wenden wir auf die zweite Gleichung die Abbildung $g_{l}$ an, erhalten
    wir
    \begin{align*}
      (g_{l}\circ(f\circ g_{r}))(y) = g_{l}(y).
    \end{align*}
    Durch Umklammern ergibt sich 
    \begin{align*}
      (g_{l}\circ f)(g_{r}(y)) = g_{l}(y).
    \end{align*}
    Wegen $g_{l}\circ f = id_{X}$ erhalten wir schließlich
    $g_{r}(y) = g_{l}(y)$, was die Gleichheit der Abbildungen zeigt.
  \end{proof}
  
\end{bem}

% ---

\begin{folg}

  Es seien $X,Y$ Mengen und $f\col X \lra Y$ eine Abbildung. Dann sind die
  folgenden Aussagen äquivalent.

  \begin{enumerate}
  \item $f$ ist bijektiv.
  \item Es gibt eine Abbildung $g\col Y \lra X$ mit $g\circ f = \id_{X}$
    und $f\circ g = \id_{Y}$.
  \end{enumerate}

  Diese eindeutige Abbildung zu $f$ nennen wir \emph{die Umkehrabbildung}
  zu $f$ und bezeichnen\footnote{Dies ist die gleiche Schreibweise, wie die
    der Urbildabbildung. Da aber die Urbildabbildung
    $f^{-1}\col \Pot{Y} \lra \Pot{Y}$ und die Umkehrabbildung
    $f^{-1}\col Y \lra X$ unterschiedliche Werte- und Bildmengen besitzen,
    besteht im Regelfall keine Verwechslungsgefahr.} diese mit $f^{-1}$.
  
\end{folg}


