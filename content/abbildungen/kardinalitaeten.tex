\section{Kardinalitäten von Mengen}

% ---

Wir wollen in diesem Abschnitt Mengen strukturell erfassen und dafür der Begriff
der \emph{Kardinalität} oder Mächtigkeit einer Menge einführen. Für viele
Aussagen der Mengentheorie ist es unwichtig, welche Elemente eine Menge
tatsächlich besitzt. Es genügt die Mengen anhand der Anzahl ihrer Elemente zu
unterscheiden.

Wir haben also vor, die unüberschaubare Vielzahl aller Mengen zu reduzieren und
Mengen mit gleicher Anzahl an Elementen zusammenzufassen. Wer sich hier an die
Motivation der GEZ aus dem ersten Kapitel erinnert, der tut dies nicht zu
unrecht. Wir werden eine geeignete Äquivalenzrelation konstruieren, die uns dank
Satz \ref{satz:part} eine Partition der Mengen liefert\footnote{An dieser Stelle
  ist es problematisch, die in Kapitel \ref{sec:mengen} eingeführte naive
  Mengenlehre zu verwenden. Um tatsächlich eine Partition der Mengen definieren
  zu können, müsste dies auf der \emph{Menge aller Mengen} geschehen, was uns
  direkt in die Russellsche Antinomie stürzen lässt. Streng genommen, müssten
  wir nicht die Menge aller Mengen, sondern die \emph{Klasse} aller Mengen
  verwenden und dafür eine axiomatische Mengenlehre einführen. Das wäre in
  diesem Rahmen weder möglich noch lehrreich.}.

% ---

\section{Die Kardinalitätsrelation}

% ---

