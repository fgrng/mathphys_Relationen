\section{Abbildungen}

Im letzten Kapitel haben wir den Begriff der \emph{Relation} eingeführt,
sind aber schnell zum Studium der spezielleren Äquivalenzrelationen
übergegangen. Wir wollen uns in diesem Kapitel mit Abbildungen (oder
Funktionen) beschäftigen. Um diese jedoch mathematisch korrekt formulieren
zu können, brauchen wir einen allgemeineren Begriff der \emph{Relation},
denn eine Abbildung setzt Elemente \emph{verschiedener} Mengen in
Beziehung.

\begin{defin}
  
  Seien $X,Y$ Mengen und $R \sbeq X\times Y$. Dann nennen wir $R$ eine
  \emph{Relation zwischen $X$ und $Y$}.

\end{defin}

Auch hier schreiben wir für $(x,y) \in R$ einfach $x \sim y$, falls keine
Verwechlungsgefahr besteht. Inbesondere sprechen wir nun von allgemeinen
Relationen und nicht von Äquivalenzrelationen. Die hier betrachteten
Relationen müssen nicht länger reflexiv, symmetrisch oder transitiv sein
und können diese Eigenschaften meist auch nich haben.

% ---

\begin{defin}
  
  Seien $X,Y$ Mengen und $R \sbeq X\times Y$ eine Relation zwischen $X$ und
  $Y$. Wir nennen $R$ …
  \begin{itemize}
    \item \emph{rechtstotal}, falls für jedes Element $y\in Y$ ein
      $x\in X$ existiert, sodass $x \sim y$.
    \item \emph{linkstotal}, falls für jedes Element $x\in X$ ein
      $y\in Y$ existiert, sodass $x \sim y$.
  \end{itemize}
  
\end{defin}

% ---

\begin{defin}
  
  Seien $X,Y$ Mengen und $R \sbeq X\times Y$ eine Relation zwischen $X$ und
  $Y$. Wir nennen $R$ …
  \begin{itemize}
    \item \emph{rechtseindeutig}, falls zu jedem $x\in X$ höchstens ein $y\in
      Y$ existiert, sodass $x\sim y$.
    \item \emph{linkseindeutig}, falls zu jedem $y\in Y$ höchstens ein
      $x\in X$ existiert, sodass $x\sim y$
  \end{itemize}
  
\end{defin}

% ---

\subsection{Die „richtige“ Definition}

% ---

Aus der Schule kennen wir den folgenden Begriff einer Abbildung

\begin{definn}
  
  Seien $X,Y$ Mengen. Eine \emph{Abbildung} $f$ ist eine
  Zuordnungsvorschrift, die jedem Element aus $X$ genau ein Element aus $Y$
  zuordnet.

\end{definn}

Schrecklich, nicht wahr? Was soll denn eine Zuordnungsvorschrift sein?
Diese Definition ist total schwammig. Zum Glück können wir bereits mit
unseren Mitteln einen exakten Abbildungsbegriff definieren.

\begin{defin}

  Seien $X,Y$ Mengen und $G \sbeq X\times Y$ eine Relation zwischen $X$ und
  $Y$. Ist $G$ linkstotal und rechtseindeutig, so nennen wir das Tripel
  $f\ceq (X,Y,G)$ eine \emph{Abbildung von $X$ nach $Y$} und schreiben
  \begin{align*}
    f \col X \lra Y.
  \end{align*}
  Es gilt also:
  \begin{itemize}
    \item zu jedem $x\in X$ existiert ein $y \in Y$ mit $x \sim y$
    \item und dieses $y$ ist das einzige Element in $Y$ mit $x \sim y$.
  \end{itemize}
  Dieses zu $x$ eindeutige $y$ mit $x\sim y$ bezeichnen wir mit
  $f(x)$ und schreiben
  \begin{align*}
    f(x) = y \,\text{ oder }\, f\col x \lmt y.
  \end{align*}
  Gegebenfalls nennen wir $X$ den \emph{Definitionsbereich}, $Y$ die
  \emph{Zielmenge} und $G$ den \emph{Graph} der Abbildung $f$.
  
\end{defin}

% ---

Dies ist die „richtige“ Definition des Abbildungsbegriffs. Alle
Grundbegriffe, die wir in der Definition verwendet haben, sind bekannt und
eindeutig definiert.

% ---

\begin{defin}

  Seien $X,Y,A,B$ Mengen, sowie $f\col X \lra Y$, $g\col A\lra B$
  Abbildungen. Dann definieren wir die \emph{Gleichheit} von $f$ und $g$
  via
  \begin{align*}
    f = g \quad :\gdw \; \begin{cases}
      & X = A \\
      \text{ und }  & Y = B \\
      \text{ und }  & f(x) = g (x) \quad \text{für alle } x\in X = A
    \end{cases}
  \end{align*}

\end{defin}

% ---

\begin{defin}

  Seien $X,Y$ Mengen und $f\col X \lra Y$ eine Abbildung. Dann induziert $f$
  zwei weitere Abbildungen zwischen den Potenzmengen $\Pot(X),\Pot(Y)$:
  \begin{align*}
    f \col \Pot(X) &\lra \Pot(Y) \\
    A &\lmt \{ f(a) \in Y \mid a\in A \}
    \intertext{die \emph{Bildabbildung} zu $f$ und}
    f^{-1} \col \Pot(Y) &\lra \Pot(X) \\
    B &\lmt \{ x \in X \mid f(x)\in B \}
  \end{align*}
  die \emph{Urbildabbildung} zu $f$. Wir nennen $f(X)$ das \emph{Bild von
    $f$} und schreiben auch $\Im(f)$.
  
\end{defin}


% ---

\begin{defin}

  Seien $X,Y,Z$ Mengen und $f\col X \lra Y$, sowie $g\col Y \lra Z$
  Abbildungen. Dann definieren wir die Abbildung
  \begin{align*}
    g \circ f \col X &\lra Z \\
    x &\lmt g(f(x))
  \end{align*}
  (gelesen: „$g$ nach $f$“) als die \emph{Komposition} oder
  \emph{Verkettung} von $f$ und $g$.

\end{defin}

% ---

\subsection{Eigenschaften von Abbildungen}

% ---

Abbildungen alleine sind recht langweilig. Wir definieren daher einige
wichtige Adjektive für Mengen und untersuchen diese neuen Begriffe.

% ---

\begin{defin}

  Seien $X,Y$ Mengen und $f\col X \lra Y$ eine Abbildung. 
  
  \begin{enumerate}
    \item Wir nennen $f$ \emph{injektiv} (oder eineindeutig), falls
      für alle $x_{1},x_{2}\in X$ mit $x_{1} \neq x_{2}$ auch $f(x_{1}) \neq
      f(x_{2})$ gilt.
    \item Wir nennen $f$ \emph{surjektiv}, falls für alle $y\in Y$
      ein $x\in X$ existiert, mit $f(x)=y$.
    \item Wir nennen $f$ \emph{bijektiv}, falls $f$ injektiv und
      surjektiv ist.
  \end{enumerate}

\end{defin}

% ---

\begin{defin}

  Seien $A,X$ Mengen mit $A\sbeq X$. Dann nennen wir
  \begin{align*}
    \iota\col A &\lra X \\
    a &\lmt a
  \end{align*}
  die \emph{natürliche Inklusion} von $A$ in $X$
\end{defin}

\begin{anm}
  $\iota$ ist eine injektive Abbildung.
\end{anm}

% ---

\begin{defin}

  Sei $X$ eine Menge und $\sim$ ein Äquivalenzrelation auf $X$. Dann nennen
  wir
  \begin{align*}
    \pi \col X &\lra \quotspace{X}{\sim} \\
    x &\lmt [x]
  \end{align*}
  die \emph{natürliche Projektion} bzgl. $\sim$.
\end{defin}

% ---

\begin{anm}

  $\pi$ ist eine surjektive Abbildung.

\end{anm}

% ---

\begin{defin}

  Sei $X$ eine Menge. Dann nennen wir
  \begin{align*}
    \id_{X}: X &\lra X \\
    x &\lmt x
  \end{align*}
  die \emph{Identität} auf $X$.

\end{defin}

% ---

\begin{anm}

  $\id_{X}$ ist eine bijektive Abbildung.

\end{anm}

% ---

Diese eben definierten Eigenschaften von Abbildungen, tauchen in der
Mathematik immer wieder auf. Um diese Eigenschaften nachzuweisen, gibt es
viele verschiedene Möglichkeiten. Die Gültigkeit einiger dieser
Möglichkeiten werden wir nachfolgend zeigen.

% ---

\begin{bem}


  Seien $X, Y$ Mengen und $f\col X \lra Y$ eine Abbildung. Dann sind
  äquivalent
  \begin{enumerate}

    \item $f$ ist injektiv, d.h. für alle $x_{1}\neq x_{2} \in X$ gilt
      $f(x_{1})\neq f(x_{2})$.

    \item Für alle $x_{1},x_{2}\in X$ mit $f(x_{1})=f(x_{2})$ gilt
      bereits $x_{1}=x_{2}$.

    \item Für alle $y\in Y$ ist $f^{-1}(\{ y \})$ höchstens
      einelementig.

    \item Der Graph $G$ von $f$ ist eine linkseindeutige Relation.

  \end{enumerate}

  \begin{proof}
    \quad

    \begin{itemize}
    \item[\tiny{(i) $\Ra$ (ii)}] 
      Dies folgt direkt durch Kontradiktion.

    \item[\tiny{(ii) $\Ra$ (iii)}] Sei $y\in Y$, $f^{-1}(\{ y \}) \neq
      \emptyset$ und $x_{1},x_{2}\in f^{-1}(\{ y \})$. Dann gilt $f(x_{1})
      = y = f(x_{2})$, also wegen $(ii)$ bereits $x_{1}=x_{2}$. Also ist
      $f^{-1}(\{ y \}) = \{ x_{1} \}$.

    \item[\tiny{(iii) $\Ra$ (iv)}] Seien $y\in Y$ und $x_{1},x_{2}\in X$
      mit $x_{1}\sim_{G}y$ und $x_{2}\sim_{G}y$. Dies bedeutet gerade, dass
      $y = f(x_{1}) = f(x_{2})$. Also gilt $x_{1},x_{2}\in f^{-1}(\{
      y\})$. Wegen $(iii)$ gilt dann $x_{1}=x_{2}$. Es ist demnach
      $\sim_{G}$ linkseindeutig.

    \item[\tiny{(iv) $\Ra$ (i)}] Seien $x_{1}, x_{2} \in X$ mit $x_{1}\neq
      x_{2}$. Angenommen es gilt $f(x_{1}) = f(x_{2}) \qec y$. Dann gilt
      aber $x_{1}\sim_{G}y$ und $x_{2}\sim y$. Dies ist ein Widerspruch zur
      Linkseindeutigkeit von $\sim_{G}$.

    \end{itemize}

  \end{proof}

\end{bem}

% ---

\begin{bem}

  Seien $X, Y$ Mengen und $f\col X \lra Y$ eine Abbildung. Dann sind
  äquivalent

  \begin{enumerate}

    \item $f$ ist surjektiv, d.h. für alle $y\in Y$ existiert ein $x\in
      X$, sodass $f(x)=y$.

    \item Für alle $y\in Y$ ist $f^{-1}(\{ y \})$ mindestens
      einelementig.

    \item Es gilt $f(X) = Y$.

    \item Der Graph $G$ von $f$ ist rechtstotal.

  \end{enumerate}

  \begin{proof}
    \quad

    \begin{itemize}
    
      \item[\tiny{(i) $\Ra$ (ii)}] 
        Sei $y\in Y$, dann existiert ein $x\in X$ mit $f(x)=y$. Also ist
        $x\in f^{-1}(\{ y \})$, also $f^{-1}(\{ y \})$ mindestens
        einelementig.

      \item[\tiny{(ii) $\Ra$ (iii)}] Sei $y\in Y$, dann ist $f^{-1}(\{ y
        \})$ mindestens einelementig. Also existiert ein $x\in X$ mit $x\in
        f^{-1}(\{ y \})$, d.h. $f(x) = y$. Also ist $y \in f(X)$, was
        „$\speq$“ zeigt. Die Inklusion „$\sbeq$“ gilt nach Definition der
        Bildabbildung.

      \item[\tiny{(iii) $\Ra$ (iv)}]
        Sei $y\in Y$. Wir zeigen die Existens von einem Element $x\in X$,
        mit $x\sim_{G} y$. Da $f(X) = Y$ gilt $y\in f(X)$, also existiert
        ein $x\in X$ mit $f(x)= y$. Mit anderen Worten $x\sim_{G}y$.

      \item[\tiny{(iii) $\Ra$ (i)}] 
        Sei $y\in Y$. Da $\sim_{G}$ rechtstotal ist, existiert ein $x\in X$
        mit $x\sim_{G} y$. Anders formuliert: $f(x) = y$.

    \end{itemize}
    
  \end{proof}

\end{bem}

% ---

\subsection{Eine erste Strukturaussage}

% ---

Wir werden an dieser Stelle eine Strukturaussage über Abbildungen
zeigen. In der Tat lässt sich jede Abbildung als Verkettung einer
injektiven, einer surjektiven und einer bijektiven Abbildung schreiben. Um
diese Aussage zu beweisen, wenden wir sämtliche Erkenntnisse aus diesem
Vortrag mehr oder weniger explizit an.

% ---

\begin{bem}
\label{bem:niveau}

  Seien $X,Y$ Mengen und $f\col X \lra Y$ eine Abbildung. Dann definiert
  \begin{align*}
    x_{1} \sim x_{2} \quad :\gdw \quad f(x_{1}) = f(x_{2})
  \end{align*}
  eine Äquivalenzrelation auf $X$.
  \begin{proof}
    Dies folgt sehr einfach aus den Eigenschaften der „$=$“-Relation.
  \end{proof}
\end{bem}

% ---

\begin{satz}

  Seien $X,Y$ Mengen und $f\col X \lra Y$ eine Abbildung. Wir betrachten das
  Diagramm

  \begin{center}
    \begin{tikzpicture}
      \matrix(m)[
      matrix of math nodes,
      row sep=4em,
      column sep=4em,
      text height=2ex, 
      text depth=0.5ex
      ]
      { 
        X & Y \\
        \quotspace{X}{\sim} & \Im(f) \\
      };
      \path[->,font=\scriptsize,>=angle 90]
      (m-1-1) edge node[above] {$f$} (m-1-2)
      (m-1-1) edge node[left] {$\pi$} (m-2-1)
      (m-2-1) edge node[below] {$\varphi$} (m-2-2)
      (m-2-2) edge node[right] {$\iota$} (m-1-2);
    \end{tikzpicture}
  \end{center}
  
  \noindent wobei $\pi\col x \mt [x]$ die Projektion bzgl. der Relation aus
  [\ref{bem:niveau}] ist und $\iota$ die natürliche Inklusion von
  $f(X)\sbeq Y$ ist. Weiter sei
  \begin{align*} 
    \varphi\col \quotspace{X}{\sim} &\lra \Im(f) \\
    [x] &\lmt f(x)
  \end{align*}
  Dann gilt
  \begin{enumerate}
    \item $\varphi$ ist wohldefiniert.
    \item $\varphi$ ist bijektiv.
    \item Das Diagramm kommutiert in dem Sinne, dass $f = \iota\circ \varphi
      \circ \pi$ gilt.
  \end{enumerate}

  \begin{proof}
  \quad
  
  \begin{enumerate}

    \item Seien $x_{1},x_{2}\in X$, mit $x_{1}\sim x_{2}$. Dann gilt
      nach Definition von $\sim$, dass $f(x_{1}) = f(x_{2})$. Also
      \begin{align*}
        \varphi\big([x_{1}]\big) = f(x_{1}) = f(x_{2}) = \varphi\big([x_{2}]\big).
      \end{align*}

    \item Wir zeigen die Injektivität und die Surjektivität von
      $\varphi$.
      \begin{itemize}

        \item[\tiny{(inj.)}] Seien $x_{1},x_{2}\in X$ mit
          $\varphi\big([x_{1}]\big) = \varphi\big([x_{2}]\big)$. Dies bedeutet
          $f(x_{1}) = f(x_{2})$ und damit ist $x_{1}\sim x_{2}$. Nach
          Bemerkung [\ref{bem:vertreter}] ist dann $[x_{1}] = [x_{2}]$.
        
        \item[\tiny{(surj.)}] Sei $y\in f(X)$. Dann existiert nach
          Definition von $f(X)$ ein $x\in X$ mit $f(x) = y$. Betrachten wir
          die Äquivalenzklasse von $x$, dann sehen wir
          $\varphi\big([x]\big) = f(x) = y$.

      \end{itemize}

    \item Wir zeigen die Gleichheit der Abbildungen
      \begin{align*}
        f &\col X \lra Y \\
        \iota\circ\varphi\circ\pi &\col X \lra Y
      \end{align*}
      Diese Abbildungen haben offensichtlich die selben Definitionsbereiche
      und Zielmengen. Bleibt noch zu zeigen, dass sie punktweise
      übereinstimmen. Sei also $x\in X$. Wir berechnen
      \begin{align*}
        (\iota\circ\varphi\circ\pi)(x) = (\iota\circ\varphi)(\pi(x)) =
        (\iota\circ\varphi)\big([x]\big) = \iota\Big(\varphi\big([x]\big)\Big) =
        \iota(f(x)) = f(x).
      \end{align*}
    \end{enumerate}
    Damit ist alles gezeigt.
    
  \end{proof}

\end{satz}


% ---