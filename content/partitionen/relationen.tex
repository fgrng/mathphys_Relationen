\section{Partitionen und Äquivalenzrelationen}

\subsection{Partitionen -- „Wohngemeinschaften und die GEZ“}

Wir blicken zurück auf den Beginn des Wintersemesters 2012/13. Viele
Studienanfänger*innen lauschen gespannt dem mathematischen Vorkurs, lernen
sich und die Stadt kennen und ziehen vielleicht in Wohngemeinschaft
zusammen. Ein paar Tage nach dem Einzug erhalten sie auf einmal Post von
der Gebühreneinzugszentrale der öffentlich-rechtlichen Rundfunkanstalten
(kurz GEZ). Seit langer Zeit verlangte sie von jedem Menschen in
Deutschland Rundfunkgebühren. Das Gebührenmodell betrachtet dazu die Menge
\begin{align*}
  D \ceq \{ x \mid x \text{ ist Mensch in Deutschland} \}
\end{align*}
Seit dem Jahr 2012 sieht die Welt aber anders aus. Die GEZ heißt jetzt ARD
ZDF Deutschlandradio Beitragsservice, die Rundfunkgebühren heißen
Rundfunkbeiträge, und das Gebührenmodell sieht eine Haushaltspauschale
vor. Wir betrachten also
\begin{align*}
  H \ceq \{ h \mid h \text{ ist ein Haushalt in Deutschland} \}
\end{align*}
Die Haushalte $h$ enthalten Menschen aus Deutschland\footnote{Die Haushalte
  $h$ können also als Mengen interpretiert werden, auch wenn wir hier
  kleine Buchstaben verwenden.}. Wir können nur spekulieren, was sich die
Zuständigen dabei gedacht haben, aber vermutlich hatten sie folgende
Hoffnungen

\begin{itemize}
  \item Der Verwaltungsaufwand wird kleiner. „$H$ ist kleiner als $D$“.
  \item Jeder Mensch ist Mitglied in einem Haushalt. Alle werden erfasst.
  \item Niemand ist Mitglied in zwei Haushalten. Keiner muss doppelt zahlen.
\end{itemize}

Die GEZ hofft also, dass $H$ eine \emph{Partition} von $D$ ist. Dieses
Begriff können wir auf mathematische Weise formulieren.

\begin{defin}
  Sei $X$ eine Menge und $P \ceq \{ P_{1}, P_{2}, P_{3}, \dots \}
  \sbeq \Pot(M)$ eine Menge von nicht-leeren Teilmengen von $X$. Wir nennen
  $P$ eine \emph{Partition von $X$}, falls
  \begin{align*}
    X = \bigcup_{i}P_{i}
  \end{align*}
  und für alle $P_{i},P_{j} \in P$ genau eine der folgenden Aussagen wahr
  ist:
  \begin{enumerate}
    \item $P_{i} = P_{j}$
    \item $P_{i} \cap P_{j} = \emptyset$
  \end{enumerate}
\end{defin}

\subsection{Äquivalenzrelationen}

Die Denkweise des Gebührenmodells der GEZ entspricht nicht unserem
Denkmuster im Alltag. Wir nehmen unser Leben in Wohngemeinschaften selten
als eine Partition aller Menschen wahr und denken wesentlich lokaler. Für
uns ist das Zusammenwohnen eher eine \emph{Beziehung} von einem Menschen zu
einem anderen. Fragen wir Alice, wo Bob denn wohnt, hören wir „Bob? Mit dem
wohne ich zusammen.“ und nicht „Bob und ich sind im selben Haushalt.“

Welche (intuitiven) Anforderungen stellen wir an eine solche Beziehung?

\begin{itemize}
  \item Alice wohnt zusammen mit Alice. Klingt komisch, ist aber so.
  \item Alice wohnt zusammen mit Bob. Also wohnt Bob auch zusammen mit
    Alice.
  \item Alice wohnt zusammen mit Bob. Bob wohnt zusammen mit mit
    Charlie. Also wohnt Alice auch zusammen mit Charlie.
\end{itemize}

% ---

\begin{defin}

  Sei $X$ eine Menge und $R \sbeq X\times X$. Dann nennen wir $R$ eine
  \emph{Relation} auf $X$. Ist $(x,y)\in R$, so schreiben wir auch
  \begin{align*}
    % x R y \,\text{ oder }\, 
    x \sim_{R} y
  \end{align*}
  und sagen \emph{$x$ steht in Relation $R$ zu $y$}.

\end{defin}

% ---

Für den Relationenbegriff kodieren wir die Elemente, die in Beziehung
stehen sollen in einem Paar, also ein Element des kartesischen
Produkts. Beziehungen, die die obigen gewünschten Eigenschften besitzen,
nennen wir Äquivalenzrelationen.

% ---

\begin{defin}

  Sei $X$ eine Menge und $R$ eine Relation auf $X$. Dann nennen wir $R$
  eine \emph{Äquivalenzrelation}, falls für alle $x,y,z\in R$ folgendes
  gilt
  \begin{enumerate}
    \item $(x,x)\in R$. \hfill (Reflexivität)
    \item Aus $(x,y)\in R$ folgt $(y,x)\in R$. \hfill (Symmetrie)
    \item Aus $(x,y),(y,z)\in R$ folgt $(x,z)\in R$. \hfill (Transitivität)
  \end{enumerate}
    
\end{defin}

% ---

Im Folgenden schreiben wir kurz $x\sim y$, falls keine Verwechslungsgefahr
mit anderen Relationen besteht. Ausserdem bezeichnen wir die Relation $R$
kurz mit $\sim$.

Wir wollen nun untersuchen, ob die beiden Denkweisen der Gebührenzentrale
und der WG-Bewohner*innen äquivalent sind oder nicht. Dazu arbeiten wir
mit den mathematischen Begriffen der Partition und der
Äquivalenzrelation. Wir betrachten zunächst die Menge der Bewohner*innen
einer WG, also die Menge aller Menschen, die zu einem bestimmten Menschen
in der „…wohnt zusammen mit…“-Relation stehen. Diese Mengen nennen wir
Äquivalenzklassen.

% ---

\begin{defin}

  Sei $X$ eine Menge, $R\sbeq X\times X$ eine Äquivalenzrelation und $x\in
  X$ ein Element in $X$. Dann nennen wir
  \begin{align*}
    [ x ] \ceq \{ y\in X \mid y\sim_{R} x \}
  \end{align*}
  die \emph{Äquivalenzklasse von $x$}. Sie besteht aus allen Elementen, die
  mit $x$ in Beziehung stehen.

\end{defin}

% ---

In unserem Beispiel der „…wohnt zusammen mit…“-Relation ist
$[ \text{Bob} ]$ die Menge aller Mitbewohner von Bob (und auch Bob). Diese
Menge ist aber identisch mit $[ \text{Alice} ]$, da Alice und Bob natürlich
die selben Mitbewohner haben. Die Auswahl des Menschen, die die
Vertreter*in der WG ist, ist beliebig und ändert nichts an der WG.

% ---

\begin{bem}
\label{bem:vertreter}

  Sei $X$ eine Menge, $\sim$ eine Äquivalenzrelation auf $X$ und $x\in
  X$. Dann gilt für alle $y \in [ x ]$:
  \begin{align*}
    [ x ] =  [ y ].
  \end{align*}
  
  \begin{proof}

    Sei $y\in [ x ]$, das heißt $y\sim x$. Da $\sim$ symmetrisch ist,
    gilt auch $x\sim y$. Wir zeigen die Gleichheit von Mengen:

    \begin{itemize}

      \item[„$\sbeq$“] Sei $z \in [ x ]$, also $z\sim x$. Wegen der
        Transitivität von $\sim$ gilt dann $z\sim y$. Also $z \in [ y ]$.

      \item[„$\speq$“] Sei $z \in [ y ]$, also $z\sim y$. Wir nutzen wieder
        die Transitivität von $\sim$ und erhalten $z\sim x$, also
        $z\in [ x ]$.

    \end{itemize}
    
    \noindent Also gilt $[ x ] \sbeq  [ y ]$ und $[ x ] \speq  [ y ]$ und
    somit $[ x ] = [ y ]$.

  \end{proof}
  
\end{bem}

% ---

Die Äquivalenzklassen von Elementen, die in Relation stehen sind demnach
bereits gleich. Wir klären nun, wie sich Äquivalenzklassen von Elementen
verhalten, die \emph{nicht} in Relation stehen.

% ---

\begin{bem}
\label{bem:disjunkt}

  Seien $X$ eine Menge, $\sim$ eine Äquivalenzrelation auf $X$ und $x,y\in
  X$. Dann gilt genau eine der folgenden Aussagen:

  \begin{enumerate}
    \item $[x] = [y]$
    \item $[x] \cap [y] = \emptyset$
  \end{enumerate}

  \begin{proof}

    Wir unterscheiden die Fälle $x\in [y]$ und $x\notin [y]$.
    \begin{itemize}

    \item[\tiny{$\big(x\in {[y]}\big)$}] Gelte $x\in [y]$, dann folgt wegen
      Bemerkung [\ref{bem:vertreter}] bereits $[x]=[y]$. Die erste Aussage
      ist also wahr. Außerdem gilt $x\in [x] \cap [y] \neq \emptyset$,
      wonach die zweite Aussage falsch ist.

    \item[\tiny{$\big(x\notin {[y]}\big)$}] Gelte $x\notin [y]$, also
      insbesondere $[x]\neq [y]$. Angenommen es existiert ein
      $z\in [x]\cap [y]$, also $z \sim x$ und $z \sim y$. Da $\sim$
      symmetrisch ist, folgt $x\sim z$ und wegen der Transitivität von
      $\sim$ auch $x\sim y$. Dann gilt aber $x\in [y]$, ein
      Widerspruch. Die Annahme war falsch und es gilt
      $[x] \cap [y] \neq \emptyset$.

    \end{itemize}
    
  \end{proof}
  
\end{bem}

% ---

\begin{defin}

  Sei $X$ eine Menge und $\sim$ eine Äquivalenzrelation auf $X$. Dann bezeichnen
  wir mit
  \begin{align*}
    \quotspace{X}{\sim} \ceq \big\{ [x] \mid x\in X \big\}
  \end{align*}
  die \emph{Menge der Äquivalenzklassen von $X$ bzgl. $\sim$}.
  

\end{defin}

% ---

Die Menge $\quotspace{X}{\sim}$ ist in unserem Beispiel die Menge aller WGs
in Deutschland. Diese Menge bildet eine Partition aller Menschen in
Deutschland und zwar genau die Partition, die der Denkweise der
Gebührenzentrale entspricht.

% ---

\begin{satz}
\label{satz:part}

  Sei $X$ eine Menge und $\sim$ eine Äquivalenzrelation auf $X$. Dann ist
  $\quotspace{X}{\sim}$ eine Partition von $M$.

  \begin{proof}

    Seien $[x_{1}],[x_{2}], [x_{3}], \dots$ die verschiedenen
    Äquivalenzklassen bezüglich der Äquivalenzrelation $\sim$. Wir müssen
    zeigen

    \begin{enumerate}
      \item $X = \bigcup_{i} [x_{i}]$
      \item Für alle $x,y \in X$ mit $[x]\neq [y]$ gilt $[x] \cap [y]
        = \emptyset$.
    \end{enumerate}

    Die zweite Aussage (ii) folgt direkt aus Bemerkung
    [\ref{bem:disjunkt}]. Wir zeigen noch (i) und müssen die Gleichheit
    dieser Mengen nachweisen. Sei $x\in X$, dann ist sicher $x\in [x]$. Die
    Äquivalenzklasse $[x]$ stimmt wegen Bemerkung [\ref{bem:disjunkt}] mit
    einer der Äquivalenzklassen $[x_{j}]$ überein. Damit ist
    \begin{align*}
      x\in [x_{j}] \sbeq \bigcup_{i} [x_{i}].
    \end{align*}
    Ausserdem ist klar, dass $\bigcup_{i}[x_{i}]
    \sbeq X$, da $[x_{i}]\sbeq X$.
    
  \end{proof}
 
\end{satz}

% ---

\begin{satz}

  Sei $X$ eine Menge und $P\sbeq \Pot(X)$ eine Partition von $X$, dann
  existiert eine Äquivalenzrelation $\sim$ auf $X$, sodass
  \begin{align*}
    \quotspace{X}{\sim} = P.
  \end{align*}

  \begin{proof}
    Diesen Beweis überlassen wir als Übungsaufgabe.
  \end{proof}
  
\end{satz}

% ---
