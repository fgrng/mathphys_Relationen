\subsection{Abbildungen}

% ---

\begin{aufg}

  Betrachte $A\ceq\{1,2,3,4\}$ und $B\ceq\{a,b,c,d\}$. Welche der folgenden
  Relationen $G_{i}\sbeq (A\times B)$ sind Abbildungen? Welche sind links- oder
  rechtstotal? Welche sind links- oder rechtseindeutig?
  \begin{enumerate}
    \item $G_{1}\ceq \{ (1,b) , (2,c), (3,d), (1,a), (4,b) \}$
    \item $G_{2}\ceq \{ (1,c) , (2,c), (3,c), (4,c) \}$
    \item $G_{3}\ceq \{ (4,b) , (2,a), (1,a) \}$
    \item $G_{4}\ceq \{ (1,a) , (2,b), (3,c), (4,d) \}$
  \end{enumerate}
  
\end{aufg}

% ---

\begin{aufg}
  Gegeben sei die Abbildung
  \begin{align*}
    f\col\RR&\lra\RR \\
    x&\lmt x^{2}.
  \end{align*}
  Werte die Bildabbidung von $f$ aus und zwar an\dots
  \begin{enumerate}
    \item $\{ 1,2,3 \}$
    \item $\{ -3,-5,4 \}$
    \item $\{ -6, 6, \sqrt{36} \}$
    \item $\ZZ$
  \end{enumerate}
  Werte die Urbildabbildung von $f$ aus und zwar an\dots
  \begin{enumerate}
    \item $\{ 1,2,3 \}$
    \item $\{ -1,-2 \}$
    \item $\{ -1,0,1 \}$
    \item $\{ 36 \}$
    \item $\emptyset$
  \end{enumerate}
\end{aufg}

% ---

\begin{aufg}
  Welche der folgenden Abbildungen $f_{i}$ sind injektiv, surjektiv, bijektiv?
  \begin{enumerate}
    \item $f_{1}\col\ZZ \lra \ZZ;\quad x \lmt |x|$
    \item $f_{2}\col\RR\setminus\{1c\} \lra \RR;\quad x \lmt \frac{1}{x - 1}$
    \item $f_{3}\col\ZZ \lra \NN_{0};\quad x \lmt |x|$
    \item $f_{4}\col\NN \lra \NN_{0}; \quad x \lmt x-1$
    \item $f_{5}\col(\RR\times\RR) \lra (\RR\times\RR);\quad (x,y) \lmt (y,x+y)$
    \item $f_{6}\col(\QQ\times\QQ) \lra \ZZ;\quad
      \left(\frac{p_{1}}{q_{1}},\frac{p_{2}}{q_{2}}\right) \lmt p_{1}q_{2}
      - p_{2}q_{1}$\\
      Was ist $f_{6}^{-1}(\{0\})$?
  \end{enumerate}
\end{aufg}

% ---

\begin{aufg}

  Seien $A,B,C,D$ Mengen und 
  \begin{align*}
    f \col &A \lra B \\
    g \col &B \lra C \\
    h \col &C \lra D
  \end{align*}
  Abbildungen. Zeige, dass die Verkettung von Abbildungen assoziativ ist,
  dass also gilt
  \begin{align*}
    (f \circ g)\circ h = f \circ (g \circ h).
  \end{align*}
  
\end{aufg}

% ---

\begin{aufg}

  Seien $X,Y$ endliche Mengen und $f\col X\lra Y$ eine Abbildung. Beweise,
  dass die folgenden Aussagen wahr sind.

  \begin{enumerate}
    \item $f$ ist genau dann injektiv, wenn $\#(f(X)) = \#(X)$ gilt.
    \item Ist $f$ injektiv, so gilt $\#(f(X)) \geq \#(Y)$.
    \item Ist $f$ surjektiv, so gilt $\#(f(X)) \leq \#(H)$.
  \end{enumerate}

\end{aufg}

% ---

\begin{aufg}

  Seien $X,Y,Z$ Mengen und $f\col X\lra Y$, $g\col Y\lra Z$ Abbildungen. Zeige:
  \begin{enumerate}
    \item Sind $f$ und $g$ injektiv, dann ist auch $g\circ f$ injektiv.
    \item Sind $f$ und $g$ surjektiv, dann ist auch $g\circ f$ surjektiv.
    \item Ist $g\circ f$ injektiv, dann ist auch $f$ injektiv. Gilt das
      auch für $g$?
    \item Ist $g\circ f$ surjektiv, dann ist auch $g$ surjektiv. Gilt das
      auch für $f$?
  \end{enumerate}
  
\end{aufg}
