\section{Naive Mengenlehre}
\label{sec:mengen}

Wir verwenden in diesem Vortrag eine naive Mengenlehre und wiederholen
einige Grundlagen. Wer sich für die Probleme des naiven Mengenbegriffs
interessiert oder diese gar fürchtet, den verweise ich auf den Vortrag
„Mengen, natürlich Zahlen,
Induktion“\footnote{\href{http://johannes.uni-hd.de/vorkurs/2013/skripte/mengen/mengen.pdf}{johannes.uni-hd.de/vorkurs/2013/skripte/mengen/mengen.pdf}}
von Tim Adler im Vorkurs zum Wintersemester 2013/14. In der praktischen
Anwendung auf dem Niveau dieses Vortrags unterscheiden sich diese
Mengenbegriffe nicht.

\begin{definn}[Georg Cantor, 1895]

  „Unter einer \emph{Menge} verstehen wir jede Zusammenfassung $M$ von
  bestimmten wohlunterschiedenen Objekten $m$ unserer Anschauung oder
  unseres Denkens (welche die \emph{Elemente} von $M$ genannt werden) zu
  einem Ganzen.“ 

\end{definn}

% ---

Wir bezeichnen Mengen typischerweise mit großen lateinischen Buchstaben
$A$, $B$, $M$, $X$, $Y$. Sprechen wir von Elementen einer Menge, so
bezeichnen wir diese mit kleinen lateinischen Buchstaben $a,b,m,x,y$. Ist
$m$ ein Element von $M$, so schreiben wir
\begin{align*}
  m \in M.
\end{align*}
Ist $m$ kein Element von $M$, so schreiben wir
\begin{align*}
  m \notin M.
\end{align*}
Wir nennen die Menge, welche keine Elemente enthält, die
\emph{leere Menge} und bezeichnen diese mit $\emptyset$.

% ---

\begin{definn}

  Seien $X$, $Y$ Mengen, dann nennen wir $X$ eine \emph{Teilmenge} von $Y$,
  falls für alle Elemente $x \in X$ auch $x \in Y$ gilt. Gegebenenfalls
  schreiben wir
  \begin{align*}
    X \sbeq Y % \text{ oder } X \subset Y.
  \end{align*}
  Existiert in diesem Fall ein Element $y\in Y$, welches nicht in $X$
  enthalten ist ($y\notin X$), so nennen wir $X$ eine \emph{echte
    Teilmenge} und schreiben
  \begin{align*}
    X \subsetneq Y.
  \end{align*}

\end{definn}

% ---

\begin{definn}

  Seien $X$, $Y$ Mengen. Wir nennen $X,Y$ \emph{gleich} oder
  \emph{identisch} und schreiben
  \begin{align*}
    X = Y,
  \end{align*}
  falls sowohl $X\sbeq Y$, als auch $Y\sbeq X$ gilt.
\end{definn}

% ---

\begin{bspn}

  Wir definieren Mengen (und damit ihre Elemente) auf verschiedene Weisen.
  \begin{enumerate}

    \item \emph{Explizite Angabe oder Aufzählung der Elemente} \\
      Geben wir alle Elemente einer Menge explizit an, so verwenden wir
      geschwungene Klammern und trennen die Elemente durch Kommata ab.
      \begin{align*}
        \{ 1, 2, 3 \} \text{ oder } \{ \text{Alice}, \text{Bob} \}.
      \end{align*}
      Dabei beachten wir weder die Reihenfolge der Angabe noch die
      Doppelnennung etwaiger Elemente. Es gilt etwa
      \begin{align*}
        \{ 1, 2, 3 \} = \{1, 3, 2 \} = \{ 3, 2, 1, 1\}.
      \end{align*}
      Besitzt eine Menge nicht endlich viele (oder sehr viele) Elemente,
      können (oder wollen) wir ihre Elemente nicht aufzählen In diesen
      Fällen behelfen wir uns mit anderen Möglichkeiten. Ist die
      Fortsetzung der Angabe klar, so können wir Fortsetzungspunkte
      verwenden.
      \begin{align*}
        \{ 1, 2, 3, 4, … \} \text{ oder } \{ a, b, c, …, A, B, C, …\}
      \end{align*}
      
    \item \emph{Angabe durch definierende Eigenschaften} \\
      Charakerisieren wir eine Menge durch eine gemeinsame Eigenschaft $E$
      ihrer Elemente, so schreiben wir 
      \begin{align*}
        \{ x \mid E(x) \}.
      \end{align*}
      Verwenden wir für diese Charakterisierung mehrere Eigenschaften
      $E_{1},E_{2},E_{3},…$, so können wir diese mittels logischen
      Operatoren verknüpfen.
      \begin{align*}
        \{ x \mid E_{1}(x) \wedge E_{2}(x) \wedge E_{3}(x) \wedge … \}\\
        \{ x \mid E_{1}(x) \vee E_{2}(x) \vee E_{3}(x) \vee … \}
      \end{align*}
      Wollen wir etwa alle natürlichen Zahlen, die gerade sind, zu einer
      Menge zusammenfassen, so formulieren wir dies durch
      \begin{align*}
        \{ x \mid x\in\NN \wedge (x\text{ ist gerade.}) \}.
      \end{align*}

    \end{enumerate}      

    Wollen wir über ein (unbestimmtes) Element einer Menge reden, so
    verwenden wir \emph{Elementvariablen}. Es kann
    $x \in \{ \text{Alice}, \text{Bob} \}$ für Alice oder Bob stehen, ist
    aber innerhalb dieser Menge unbestimmt. Alice ist eine
    \emph{Elementkonstante} von $\{ \text{Alice}, \text{Bob} \}$, also
    bestimmt.

\end{bspn}

% ---

\begin{definn}

  Sei $X$ eine Menge. Dann nennen wir die Menge aller Teilmengen von $X$
  die \emph{Potenzmenge von $X$} und bezeichnen diese mit
  \begin{align*}
    \Pot(X) \ceq \{ Y \mid Y\sbeq X\}.
  \end{align*}

\end{definn}

% ---

\begin{bspn}

  Sei $X \ceq \{ 1, 2, 3 \}$. Dann ist
  \begin{align*}
    \Pot(X) = \big\{ \emptyset, \{  1 \}, \{ 2 \}, \{ 3 \}, 
     \{ 1,2 \}, \{ 1,3 \}, \{ 2,3 \}, 
     \{ 1,2,3 \} \big\}.
  \end{align*}

\end{bspn}

% ---

\begin{definn}

  Seien $X,Y$ Mengen, dann nennen wir die Menge aller (geordneten) Paare
  von Elementen von $X$ und $Y$ das \emph{kartesische Produkt} und
  schreiben
  \begin{align*}
    X \times Y \ceq \{ (x,y) \mid x\in X, y\in Y \}.
  \end{align*}
  
\end{definn}

% ---

\begin{bspn}

  Seien $X \ceq \{ 1, 2, 3 \}$ und $Y \ceq \{ \text{Alice}, \text{Bob}
  \}$. Dann ist
  \begin{align*}
    X \times Y = \big\{ &(1, \text{Alice}), (2, \text{Alice}), (3,\text{Alice}), \\
    &(1, \text{Bob}), (2, \text{Bob}), (3, \text{Bob}) \big\}
  \end{align*}
  
\end{bspn}


% ---

\begin{definn}

  Seien $X,Y$ und $X_{1}, X_{2}, X_{3}, \dots$ Mengen. Wir definieren
  \begin{align*}
    X \cup Y \ceq \{ z \mid z\in X \text{ oder } z\in Y\}
  \end{align*}
  als die \emph{Vereinigung von $X$ und $Y$} und
  \begin{align*}
    X \cap Y \ceq \{ z \mid z\in X \text{ und } z\in Y\}
  \end{align*}
  als den \emph{Schnitt von $X$ und $Y$}.
  Ausserdem schreiben wir
  \begin{align*}
    \bigcup_{i}X_{i} \,\text{ bzw. }\, \bigcap_{i}X_{i}
  \end{align*}
  für die Vereinigung bzw. den Schnitt aller Mengen $X_{1}, X_{2}, X_{3}, \dots$.\footnote{Diese
    Notation ist nur dann wohldefiniert, falls $\cup$ bzw. $\cap$ assoziativ
  und kommutativ sind. Der Nachweis dieser Eigenschaften ist eine lohnende
  Übungsaufgabe.}
\end{definn}

% ---
















